%\documentclass{beamer}
\documentclass[handout]{beamer}
\usetheme{Marburg}
\useoutertheme{infolines}
\newcommand{\answers}{1}

\usepackage{amsmath}
\usepackage{caption}
\usepackage{color}
\usepackage{enumerate}
\usepackage{listings}
\usepackage{hyperref}
\usepackage{mathrsfs}
\usepackage{natbib}
\usepackage{url}

\providecommand{\all}{\ \forall \ }
\providecommand{\bs}{\backslash}
\providecommand{\e}{\varepsilon}
\providecommand{\E}{\ \exists \ }
\providecommand{\lm}[2]{\lim_{#1 \rightarrow #2}}
\providecommand{\m}[1]{\mathbb{#1}}
\providecommand{\nv}{{}^{-1}}
\providecommand{\ov}[1]{\overline{#1}}
\providecommand{\p}{\newpage}
\providecommand{\q}{$\quad$ \newline}
\providecommand{\rt}{\rightarrow}
\providecommand{\Rt}{\Rightarrow}
\providecommand{\vc}[1]{\boldsymbol{#1}}
\providecommand{\wh}[1]{\widehat{#1}}

\hypersetup{colorlinks,linkcolor=,urlcolor=blue}
\numberwithin{equation}{section}

\definecolor{dkgreen}{rgb}{0,0.6,0}
\definecolor{gray}{rgb}{0.5,0.5,0.5}
\definecolor{mauve}{rgb}{0.58,0,0.82}

\lstset{ 
  language=C,                % the language of the code
  basicstyle= \tiny,           % the size of the fonts that are used for the code
  numbers=left,
  numberfirstline=true,
  numbersep=5pt,                  % how far the line-numbers are from the code
  backgroundcolor=\color{white},      % choose the background color. You must add \usepackage{color}
  showspaces=false,               % show spaces adding particular underscores
  showstringspaces=false,         % underline spaces within strings
  showtabs=false,                 % show tabs within strings adding particular underscores
  frame=lrb,                   % adds a frame around the code
  rulecolor=\color{black},        % if not set, the frame-color may be changed on line-breaks within not-black text 
  tabsize=2,                      % sets default tabsize to 2 spaces
  captionpos=t,                   % sets the caption-position 
  breaklines=true,                % sets automatic line breaking
  breakatwhitespace=false,        % sets if automatic breaks should only happen at whitespace
  %title=\lstname,                   % show the filename of files included with \lstinputlisting;
  keywordstyle=\color{blue},          % keyword style
  commentstyle=\color{gray},       % comment style
  stringstyle=\color{dkgreen},         % string literal style
  escapeinside={\%*}{*)},            % if you want to add LaTeX within your code
  morekeywords={*, ...},               % if you want to add more keywords to the set
  xleftmargin=0.053in, % left horizontal offset of caption box
  xrightmargin=-.03in % right horizontal offset of caption box
}

%\DeclareCaptionFont{white}{\color{white}}
%\DeclareCaptionFormat{listing}{\parbox{\textwidth}{\colorbox{gray}{\parbox{\textwidth}{#1#2#3}}\vskip-0.05in}}
%\captionsetup[lstlisting]{format = listing, labelfont = white, textfont = white}
%For caption-free listings, comment out the 3 lines above and uncomment the 2 lines below.
 \captionsetup{labelformat = empty, labelsep = none}
 \lstset{frame = single}

\title{Introduction to Python 2 for statisticians}
\author{Will Landau}
\date{November 18, 2013}
\institute{Iowa State University}

\begin{document}

\begin{frame}
\titlepage
 \end{frame}
 
 \begin{frame}
\frametitle{Outline}
\tableofcontents
\end{frame}
 
 \AtBeginSection[]
{
   \begin{frame}
       \frametitle{Outline}
       \tableofcontents[currentsection]
   \end{frame}
}

\section{Basic elements}

\begin{frame}[fragile]
\frametitle{Preliminaries}
\begin{itemize}
\item Python is a high-level multipurpose interpreted language. 
\pause \item It's clumsier than R at statistical number crunching, but better than R at string manipulation.
\pause \item Open the interpreter by typing {\tt python} into the command line.
\begin{lstlisting}[name=ex]
> python
Python 2.7.2 (default, Oct 11 2012, 20:14:37) 
[GCC 4.2.1 Compatible Apple Clang 4.0 (tags/Apple/clang-418.0.60)] on darwin
Type "help", "copyright", "credits" or "license" for more information.
>>> 
\end{lstlisting}
\end{itemize}
\end{frame}

\begin{frame}[fragile]
\frametitle{Preliminaries}
\begin{itemize}
\item You can create variables and do arithmetic:
\end{itemize}
\begin{lstlisting}[name=ex]
>>> a = 1
>>> b = 2
>>> c = "goober"
>>> c
'goober'
>>> a+b
3
>>> 
\end{lstlisting}
\begin{itemize}
\pause \item You can make a python script like {\tt hello\_world.py}:
\end{itemize}
\begin{lstlisting}[name=ex]
s = "Hello World"
print(s)
\end{lstlisting}
\begin{itemize}
\pause \item And run it in the command line:
\end{itemize}
\begin{lstlisting}[name=ex]
> python hello_world.py
Hello World
\end{lstlisting}
\end{frame}

\begin{frame}[fragile]
\frametitle{Preliminaries}
\begin{itemize}
\item Use the hash sign for single-line comments.
\end{itemize}

\begin{lstlisting}[name=ex]
>>> # print("Hello world!")
>>> 
\end{lstlisting}


\pause 
\begin{itemize}
\item Triple-quoted strings span multiple lines and serve as multi-line comments.
\end{itemize}

\begin{lstlisting}[name=ex]
# a.py
"""
This program
does nothing. 
"""
\end{lstlisting}

\begin{lstlisting}[name=ex]
> python a.py
>
\end{lstlisting}
\end{frame}


\begin{frame}[fragile]
\frametitle{Preliminaries}

\begin{itemize}
\item Formatting strings
\end{itemize}

\begin{lstlisting}[name=ex]
>>> s = "Today's date is {month}/{day}/{year}".format(month = 10, day = 22, \
...       year = 2012)
>>>       
... print(s)
Today's date is 10/22/2012
>>> 
\end{lstlisting}

\pause 
\begin{itemize}
\item Every string has builtin methods such as {\tt format()}, which can be accessed with the dot operator.
\pause \item There are multiple ways to format output.
\end{itemize}

\begin{lstlisting}[name=ex]
>>> a = 3
>>> b = 4.8878
>>> s = format("sample %d: mass= %0.3fg" % (a, b))
>>> print(s)
sample 3: mass= 4.888g
>>> 
>>> print("sample %d: mass= %0.3fg" % (a, b))
sample 3: mass= 4.888g
>>> 
\end{lstlisting}
\end{frame}


\begin{frame}[fragile]
\frametitle{User-defined functions}

\begin{itemize}
\item I can define and use my own function like this:
\end{itemize}

\begin{lstlisting}[name=ex]
>>> def f1(a):
...   if a == 0:
...     print("hi")
...     return(0)
...   elif a < 0:
...     print("stop")
...     return(1)
...   else:
...     return(5)
... 
>>> f1(0)
hi
0
>>> f1(1)
5
>>> f1(-1)
stop
1
>>> 
\end{lstlisting}
\end{frame}




\begin{frame}[fragile]
\frametitle{Indentation}

\begin{itemize}
\item In python, indention is used to denote nested blocks of code (like \{ and \} in C). Indentation must be consistent. 
\pause \item The following script, {\tt a.py}, has an indentation error.
\end{itemize}

\begin{lstlisting}[name=ex]
# a.py
def f1(a):
  if a == 0:
    print("hi")
    return(0)
  elif a < 0:
    print("stop")
    return(1)
  else:
  return(5)
\end{lstlisting}

\pause
\begin{itemize}
\item When I try to run it,
\end{itemize}

\begin{lstlisting}[name=ex]
> python a.py
  File "a.py", line 10
    return(5)
         ^
IndentationError: expected an indented block
\end{lstlisting}
\end{frame}







\begin{frame}[fragile]
\frametitle{Line continuation}

\begin{itemize}
\item With the exception of multi-line quotes, you have to use the line continuation character, '$\backslash$', when you want to wrap text in your code:
\end{itemize}

\begin{lstlisting}[name=ex]
>>> 1+1+1+1+1+1+1+1+1+1+1+1+1+1+1+1+1+1+1+1+1+1+1+1+1+1+1+1+1
29
>>> 1+1+1+1+1+1+1+
  File "<stdin>", line 1
    1+1+1+1+1+1+1+
                 ^
SyntaxError: invalid syntax
>>>
>>> 1+1+1+1+1+1+1+ \
... 1+1+1+1+1+1+1+1
15
>>> 
\end{lstlisting}
\end{frame}




\begin{frame}[fragile]
\frametitle{Logic and control flow}

\begin{lstlisting}[name=ex]
>>> 1 and 2
2
>>> 1 == 1
True
>>> 1 == 0
False
>>> 1 == 1 and 2 == 0
False
>>> 1 > 1 or 2 <= 5
True
>>> not True
False
>>> True and not False
True
>>> if True:
...   print("yes")
... else:
...   print("no")
... 
yes
>>> 
\end{lstlisting}
\end{frame}

\begin{frame}[fragile]
\frametitle{Logic and control flow}

\begin{lstlisting}[name=ex]
>>> a = 1
>>> if a == 2:
...   print("two")
... elif a < -1000:
...   print("a is small")
... elif a > 100 and not a % 2:
...   print("a is big and even")
... else:
...   print("a is none of the above.")
... 
a is none of the above.
>>> 
\end{lstlisting}
\end{frame}







\begin{frame}[fragile]
\frametitle{Stings: where Python is strong}

\begin{itemize}
\item You can use single, double, or triple quotes to denote string literals.
\end{itemize}

\begin{lstlisting}[name=ex]
>>> a = "Hello World"
>>> b = 'Python is groovy'
>>> c = """Computer says 'No'"""
>>> 
\end{lstlisting}

\pause 
\begin{itemize}
\item Triple quotes can extend over multiple lines, but single and double quotes cannot.
\end{itemize}

\begin{lstlisting}[name=ex]
>>> c = """Computer says 'no'
... because another computer
... says yes"""
>>> a = "hello
  File "<stdin>", line 1
    a = "hello
             ^
SyntaxError: EOL while scanning string literal
>>> 
\end{lstlisting}
\end{frame}



\begin{frame}[fragile]
\frametitle{Strings: where Python is strong}

\begin{itemize}
\item Strings are stored as sequences of characters.
\end{itemize}

\begin{lstlisting}[name=ex]
>>> a = "Hello World"
>>> a[0]
'H'
>>> a[:5]
'Hello'
>>> a[6:]
'World'
>>> a[3:8]
'lo Wo'
>>> 
\end{lstlisting}

\pause
\begin{itemize}
\item You can convert numeric types into strings and vice versa.
\end{itemize}

\begin{lstlisting}[name=ex]
>>> z = "90"
>>> z
'90'
>>> int(z)
90
>>> float(z)
90.0
>>> str(90.25)
'90.25'
>>> 
\end{lstlisting}
\end{frame}


\begin{frame}[fragile]
\frametitle{Strings: where Python is strong}

\begin{itemize}
\item And you can concatenate strings.
\end{itemize}

\begin{lstlisting}[name=ex]
>>> "123" + "abc"
'123abc'
>>> "123" + str(123.45)
'123123.45'
>>> a = 1
>>> b = "2"
>>> str(a) + b
'12'
>>> 
\end{lstlisting}
\end{frame}


\begin{frame}[fragile]
\frametitle{Strings: where Python is strong}

\begin{itemize}
\item There are several useful methods for strings.
\end{itemize}

\begin{lstlisting}[name=ex]
>>> s = "Hello world!"
>>> len(s)
12
>>>
>>> s = "5, 4, 2, 9, 8, 7, 28"
>>> s.count(",")
6
>>> s.find("9, ")
9
>>> s[9:12]
'9, '
>>> "abc123".isalpha()
False
>>> "abc123".isalnum()
True
>>> s.split(",")
['5', ' 4', ' 2', ' 9', ' 8', ' 7', ' 28']
>>> ", ".join(["ready", "set", "go"])
'ready, set, go'
>>> "ready\n set\n go".splitlines()
['ready', ' set', ' go']
>>> "ready set go".splitlines()
['ready set go']
>>> 
\end{lstlisting}
\end{frame}




\begin{frame}[fragile]
\frametitle{Lists}

\begin{itemize}
\item In python, a list is an ordered sequence of objects, each of which can have any type.
\end{itemize}

\begin{lstlisting}[name=ex]
>>> s = [1, 2, "Five!", ["Three, sir!", "Three!"]]
>>> len(s)
4
>>> 
>>> s[0:1]
[1]
>>> s[2]
'Five!'
>>> s[2][1]
'i'
>>> s[3]
['Three, sir!', 'Three!']
>>> s[3][0]
'Three, sir!'
>>> s[3][1]
'Three!'
>>> s[3][1][1]
'h'
>>> s.append("new element")
>>> s
[1, 2, 'Five!', ['Three, sir!', 'Three!'], 'new element']
\end{lstlisting}
\end{frame}


\begin{frame}[fragile]
\frametitle{Lists}

\begin{itemize}
\item I can append and remove list elements.
\end{itemize}

\begin{lstlisting}[name=ex]
>>> l = ["a", "b", "c"]
>>> l.append("d")
>>> l.append("c")
>>> l
['a', 'b', 'c', 'd', 'c']
>>> l.remove("a")
>>> l
['b', 'c', 'd', 'c']
>>> l.remove("c")
>>> l
['b', 'd', 'c']
>>> l.remove("c")
>>> l
['b', 'd']
>>> 
\end{lstlisting}
\end{frame}





\begin{frame}[fragile]
\frametitle{Tuples}

\begin{lstlisting}[name=ex]
>>> a = ()
>>> b = (3,)
>>> c = (3,4,"thousand")
>>> len(c)
3
>>>
>>> number1, number2, word = c
>>> number1
3
>>> number2
4
>>> word
'thousand'
>>> keys =["name", "status", "ID"]
>>> values = ["Joe", "approved", 23425]
>>> z = zip(keys, values)
>>> z
[('name', 'Joe'), ('status', 'approved'), ('ID', 23425)]
\end{lstlisting}
\end{frame}


\begin{frame}[fragile]
\frametitle{Dictionaries}

\begin{lstlisting}[name=ex]
>>> stock = {
... "name" : "GOOG",
... "shares" : 100,
... "price" : 490.10 }
>>> stock
{'price': 490.1, 'name': 'GOOG', 'shares': 100}
>>> stock["name"]
'GOOG'
>>> stock["date"] = "today"
>>> stock
{'date': 'today', 'price': 490.1, 'name': 'GOOG', 'shares': 100}
>>> keys = ["name", "status", "ID"]
>>> values = ["Joe", "approved", 23425]
>>> zip(keys, values)
[('name', 'Joe'), ('status', 'approved'), ('ID', 23425)]
>>> d = dict(zip(keys, values))
>>> d
{'status': 'approved', 'name': 'Joe', 'ID': 23425}
>>> 
\end{lstlisting}
\end{frame}

\begin{frame}[fragile]
\frametitle{Iteration and looping}

\begin{itemize}
\item There are many ways to iterate.
\end{itemize}

\begin{lstlisting}[name=ex]
# a.py
a = "Hello World"
# Print out the individual characters in a 
for c in a:
  print c
\end{lstlisting}

\pause 
\begin{lstlisting}[name=ex]
> python a.py
H
e
l
l
o
 
W
o
r
l
d
\end{lstlisting}
\end{frame}


\begin{frame}[fragile]
\frametitle{Iteration and looping}

\begin{lstlisting}[name=ex]
# a.py
b = ["Dave","Mark","Ann","Phil"] 
# Print out the members of a list 
for name in b:
  print name
\end{lstlisting}

\pause
\begin{lstlisting}[name=ex]
> python a.py
Dave
Mark
Ann
Phil
\end{lstlisting}

\pause
\begin{lstlisting}[name=ex]
# a.py
c = { 'GOOG' : 490.10, 'IBM' : 91.50, 'AAPL' : 123.15 } 
# Print out all of the members of a dictionary
for key in c:
  print key, c[key]
\end{lstlisting}

\pause
\begin{lstlisting}[name=ex]
> python a.py
GOOG 490.1
AAPL 123.15
IBM 91.5
\end{lstlisting}
\end{frame}

\begin{frame}[fragile]
\frametitle{Iteration and looping}

\begin{lstlisting}[name=ex]
# a.py
for n in [0, 1,2,3,4,5,6,7,8,9]:
  print("2 to the %d power is %d" % (n, 2**n))
\end{lstlisting}

\pause

\begin{lstlisting}[name=ex]
> python a.py
2 to the 0 power is 1
2 to the 1 power is 2
2 to the 2 power is 4
2 to the 3 power is 8
2 to the 4 power is 16
2 to the 5 power is 32
2 to the 6 power is 64
2 to the 7 power is 128
2 to the 8 power is 256
2 to the 9 power is 512
\end{lstlisting}
\end{frame}


\begin{frame}[fragile]
\frametitle{}

\begin{lstlisting}[name=ex]
# a.py
for n in range(9):
  print("2 to the %d power is %d" % (n, 2**n))
\end{lstlisting}

\pause

\begin{lstlisting}[name=ex]
> python a.py
2 to the 0 power is 1
2 to the 1 power is 2
2 to the 2 power is 4
2 to the 3 power is 8
2 to the 4 power is 16
2 to the 5 power is 32
2 to the 6 power is 64
2 to the 7 power is 128
2 to the 8 power is 256
2 to the 9 power is 512
\end{lstlisting}
\end{frame}



\begin{frame}[fragile]
\frametitle{{\tt range()} and {\tt xrange()}}

\begin{itemize}
\item For lengthy iterations, don't use {\tt range()} because it fully populates a list and takes up a lot of memory.
\pause \item  Instead, use {\tt xrange()}, which gives you your iteration indices on a need-to-know basis.
\end{itemize}

\pause \begin{lstlisting}[name=ex]
# a.py
x = 0
for n in xrange(999999999):
  x = x + 1
print(x)
\end{lstlisting}
\end{frame}



\begin{frame}[fragile]
\frametitle{Generators}

\begin{itemize}
\item {\tt range()} is a special case of a larger class of functions called generators.
\end{itemize}

\begin{lstlisting}[name=ex]
>>> def countdown(n):
...   print "Counting down!" 
...   while n > 0:
...     yield n # Generate a value (n) 
...     n -= 1
... 
>>> c = countdown(5)
>>> c.next()
Counting down!
5
>>> c.next()
4
>>> c.next()
3
>>> 
\end{lstlisting}
\end{frame}




\begin{frame}[fragile]
\frametitle{List comprehensions}

\begin{lstlisting}[name=ex]
>>> nums = [1, 2, 3, 4, 5]
>>> squares = [n * n for n in nums]
>>> squares
[1, 4, 9, 16, 25]
>>> a = [-3,5,2,-10,7,8]
>>> b = 'abc'
>>> [2*s for s in a]
[-6, 10, 4, -20, 14, 16]
>>> [s for s in a if s >= 0]
[5, 2, 7, 8]
>>> [(x,y) for x in a
... for y in b if x > 0 ]
[(5, 'a'), (5, 'b'), (5, 'c'), (2, 'a'), (2, 'b'), (2, 'c'), (7, 'a'), 
(7, 'b'), (7, 'c'), (8, 'a'), (8, 'b'), (8, 'c')]
>>> [(1,2), (3,4), (5,6)]
[(1, 2), (3, 4), (5, 6)]
\end{lstlisting}

\pause
\begin{itemize}
\item General syntax:

\begin{lstlisting}[name=ex]
[expression for item_1 in iterable_1 if condition_1 
            for item_2 in iterable_2 if condition_2
            ...
            for item_N in iterable_N if condition_N ]
\end{lstlisting}
\end{itemize}
\end{frame}



\begin{frame}[fragile]
\frametitle{Lambda functions, {\tt filter()}, {\tt map()}, and {\tt reduce()}}

\begin{itemize}
\item {\bf Lambda function}: a compact way of writing a function. You can think of a lambda function as a``function literal".
\pause \item {\tt filter(fun, list)}: returns a list of all the elements e in list for which fun(e) is true.
\pause \item {\tt map(fun, list)}: applies fun to each element of list and returns the result in a new list.
\pause \item {\tt reduce(fun, list)}: equivalent to the following (length of list is n).
\end{itemize}

\begin{lstlisting}[name=ex]
value = fun(list[0], list[1])
value = fun(value, list[2])
value = fun(value, list[3])
...
value = fun(value, list[n])
\end{lstlisting}
\end{frame}

\begin{frame}[fragile]
\frametitle{Lambda functions, {\tt filter()}, {\tt map()}, and {\tt reduce()}}

\begin{lstlisting}[name=ex]
>>> f = lambda x: x > 3 and x % 2 != 0
>>> f(4)
False
>>> f(5)
True
>>> f(6)
False
>>> 
>>> filter(lambda x: x > 3, [0, 1, 2, 3, 4, 5])
[4, 5]
>>>
>>>
>>> l = range(3)
>>> map(str, l)
['0', '1', '2']
>>>
>>> map(lambda x: x*x, l)
[0, 1, 4]
>>>
>>> reduce(lambda x, y: x+y, range(1, 11)) # sum the numbers 1 to 10
55
>>>
\end{lstlisting}
\end{frame}



\begin{frame}[fragile]
\frametitle{File I/O}

\begin{itemize}
\item If I run:
\end{itemize}

\begin{lstlisting}[name=ex]
# a.py
import random

f = open("data.txt", "w")
f.write("x y\n")
for i in xrange(10):
  f.write("%0.3f %0.3f\n" % (random.random(), random.random()))
\end{lstlisting}


\pause


\begin{itemize}
\item The file, {\tt data.txt}, is generated:
\end{itemize}

\begin{lstlisting}[name=ex]
x y
0.506 0.570
0.887 0.792
0.921 0.641
0.894 0.664
0.494 1.000
0.745 0.734
0.274 0.127
0.075 0.381
0.449 0.995
0.355 0.807
\end{lstlisting}
\end{frame}


\begin{frame}[fragile]
\frametitle{File I/O}

\begin{itemize}
\item I can read {\tt data.txt} with:
\end{itemize}

\begin{lstlisting}[name=ex]
>>> f = open("data.txt")
>>> header = f.readline()
>>> data = f.readlines()
>>> 
>>> header
'x y\n'
>>> data
['0.506 0.570\n', '0.887 0.792\n', '0.921 0.641\n', '0.894 0.664\n', '0.494 1.000\n', 
 '0.745 0.734\n', '0.274 0.127\n', '0.075 0.381\n', '0.449 0.995\n', '0.355 0.807\n']
>>>
>>> header = header.replace("\n","")
>>> header
'x y'
>>>
>>> d = [d.replace("\n","") for d in data]
>>> d
['0.506 0.570', '0.887 0.792', '0.921 0.641', '0.894 0.664', '0.494 1.000',
 '0.745 0.734', '0.274 0.127', '0.075 0.381', '0.449 0.995', '0.355 0.807']
>>> 
\end{lstlisting}
\end{frame}


\begin{frame}[fragile]
\frametitle{File I/O}

\begin{itemize}
\item And then I can process it into a nicer format.
\end{itemize}

\begin{lstlisting}[name=ex]
>>> data = [d.split(" ") for d in data]
>>> data
[['0.506', '0.570'], ['0.887', '0.792'], ['0.921', '0.641'], ['0.894', '0.664'], 
['0.494', '1.000'], ['0.745', '0.734'], ['0.274', '0.127'], ['0.075', '0.381'], 
['0.449', '0.995'], ['0.355', '0.807']]
>>> 
>>> data = [map(float, d) for d in data]
>>> data
[[0.506, 0.57], [0.887, 0.792], [0.921, 0.641], [0.894, 0.664], [0.494, 1.0], 
[0.745, 0.734], [0.274, 0.127], [0.075, 0.381], [0.449, 0.995], [0.355, 0.807]]
\end{lstlisting}
\end{frame}


\begin{frame}[fragile]
\frametitle{Modules}

\begin{itemize}
\item Modules are external packages of code. The are not usually builtin, but  they can be imported. 
\end{itemize}

\begin{lstlisting}[name=ex]
>>> sqrt(10)
Traceback (most recent call last):
  File "<stdin>", line 1, in <module>
NameError: name 'sqrt' is not defined
>>> 
>>> import math
>>> sqrt(10)
Traceback (most recent call last):
  File "<stdin>", line 1, in <module>
NameError: name 'sqrt' is not defined
>>> 
>>> math.sqrt(10)
3.1622776601683795
>>> 
\end{lstlisting}
\end{frame}

\begin{frame}[fragile]
\frametitle{Modules}

\begin{itemize}
\item If you don't want to write {\tt math.sqrt()} every single time you want to compute a square root, you can use a shortcut.
\end{itemize}

\begin{lstlisting}[name=ex]
>>> import math as m
>>> m.sqrt(10)
3.1622776601683795
>>> 
\end{lstlisting}

\pause

\begin{itemize}
\item Or better yet, 
\end{itemize}

\begin{lstlisting}[name=ex]
>>> from math import *
>>> sqrt(10)
3.1622776601683795
>>> 
\end{lstlisting}

\pause

\begin{itemize}
\item Where is the math module?
\end{itemize}

\begin{lstlisting}[name=ex]
>>> import math
>>> math.__file__
'/usr/lib64/python2.6/lib-dynload/mathmodule.so'
>>> 
\end{lstlisting}
\end{frame}


\begin{frame}[fragile]
\frametitle{Installing modules locally in impact1}

\begin{itemize}\scriptsize
\item We don't have permission to install modules ourselves in {\tt /usr/}. We have to either bother the STAT IT people or install our package locally.
\pause \item Suppose I want to install the SQLAlchemy module. To download it into my home directory, I work from linux command line on impact1.
\end{itemize}

\lstset{language=bash}
\pause \begin{lstlisting}[name=pkg]
> cd ~
> pwd
/home/landau
> ls
stuff
> wget http://prdownloads.sourceforge.net/sqlalchemy/SQLAlchemy-0.7.9.tar.gz?download
# output of wget ...
> ls
stuff	SQLAlchemy-0.7.9.tar.gz
> tar -zxvf SQLAlchemy-0.7.9.tar.gz 
# output of tar...
> cd SQLAlchemy-0.7.9
> python setup.py build
# output of python...
> python setup.py install --user
# output of python...
Installed /home/landau/.local/lib/python2.6/site-packages/SQLAlchemy-0.7.9-py2.6-linux-x86_64.egg
Processing dependencies for SQLAlchemy==0.7.9
Finished processing dependencies for SQLAlchemy==0.7.9
\end{lstlisting}
\end{frame}



\begin{frame}[fragile]
\frametitle{Installing modules locally in impact1}

\begin{itemize} \scriptsize
\item IMPORTANT: take note that SQLAlchemy was installed in the directory, {\tt /home/landau/.local/lib/python2.6/site-packages/}. I must ``export" this path in my {\tt .bashrc} file so that Python knows where my new module lives. 
\pause \item I move to my home directory and open {\tt .bashrc}:
\end{itemize}

\begin{lstlisting}[name=pkg]
[landau@impact1 SQLAlchemy-0.7.9]$ cd ~
[landau@impact1 ~]$ emacs .bashrc
\end{lstlisting}


\pause
\begin{itemize}
\item The file itself currently looks like this:
\end{itemize}

\begin{lstlisting}[name=bashrc]
# .bashrc                                                                                                             

# Source global definitions                                                                                           
if [ -f /etc/bashrc ]; then
        . /etc/bashrc
fi

# User specific aliases and functions
\end{lstlisting}
\end{frame}

\begin{frame}[fragile]
\frametitle{Installing modules locally in impact1}

\begin{itemize}
\item I add a couple lines to the end so that Python knows where to find my package:
\end{itemize}

\begin{lstlisting}[name=bashrc]
# .bashrc                                                                                                             

# Source global definitions                                                                                           
if [ -f /etc/bashrc ]; then
        . /etc/bashrc
fi

# User specific aliases and functions                                                                                 

export PYTHONPATH=$HOME/.local/lib/python2.6/site-packages:$PYTHONPATH
export PATH=$HOME/.local/bin:$PATH
\end{lstlisting}
\pause

\begin{itemize}
\item Once I've made the changes and I log out and in so that the changes take effect, I'm ready to import SQLAlchemy.
\end{itemize}

\begin{lstlisting}[name=ex2]
>>> import sqlalchemy
>>>
\end{lstlisting}

\pause
\begin{itemize}
\item Remember to use all lower case letters for modules in the {\tt import} statement.
\end{itemize}
\end{frame}




\begin{frame}[fragile]
\frametitle{The {\tt sys} module}

\begin{itemize}
\item {\tt sys} is a module of system-specific parameters and functions.
\end{itemize}

\begin{lstlisting}[name=ex2]
# a.py
import sys

for arg in sys.argv:
  print arg
\end{lstlisting}

\pause
\begin{lstlisting}[name=ex2]
> python a.py 1 2 3 4 5 3sir! 3!
a.py
1
2
3
4
5
3sir!
3!
\end{lstlisting}
\end{frame}


\section{The NumPy module}

\begin{frame}[fragile]
\frametitle{The {\tt NumPy} module}

\begin{itemize}
\item Important module for arrays and matrices. 
\end{itemize}

\begin{lstlisting}[name=ex2]
>>> from numpy  import *
>>> a = arange(15)
>>> a
array([ 0,  1,  2,  3,  4,  5,  6,  7,  8,  9, 10, 11, 12, 13, 14])
>>> a = a.reshape(3,5)
>>> a
array([[ 0,  1,  2,  3,  4],
       [ 5,  6,  7,  8,  9],
       [10, 11, 12, 13, 14]])
>>> a.transpose()
>>> a.transpose()
array([[ 0,  5, 10],
       [ 1,  6, 11],
       [ 2,  7, 12],
       [ 3,  8, 13],
       [ 4,  9, 14]])
>>> a.shape
(3, 5)
>>> a.size
15
>>> type(a)
<type 'numpy.ndarray'>
>>> 
>>> zeros( (3,4) )
array([[0.,  0.,  0.,  0.],
       [0.,  0.,  0.,  0.],
       [0.,  0.,  0.,  0.]])
>>>
\end{lstlisting}
\end{frame}

\begin{frame}[fragile]
\frametitle{The {\tt NumPy} module}

\begin{lstlisting}[name=ex2]
>>> ones( (2,3,4), dtype=int16 )                # dtype can also be specified
array([[[ 1, 1, 1, 1],
        [ 1, 1, 1, 1],
        [ 1, 1, 1, 1]],
       [[ 1, 1, 1, 1],
        [ 1, 1, 1, 1],
        [ 1, 1, 1, 1]]], dtype=int16)
>>>
>>> empty( (2,3) )
array([[  3.73603959e-262,   6.02658058e-154,   6.55490914e-260],
       [  5.30498948e-313,   3.14673309e-307,   1.00000000e+000]])
>>>       
>>> b = array( [ [1.5,2,3], [4,5,6] ] )
>>> b
array([[ 1.5,  2. ,  3. ],
       [ 4. ,  5. ,  6. ]])
>>> print(b)
[[ 1.5  2.   3. ]
 [ 4.   5.   6. ]]
>>>
>>> sum(b)
21.5
>>>
>>> a = array( [20,30,40,50] )
>>> b = arange( 4 )
>>> b
array([0, 1, 2, 3])
>>> c = a-b
>>> c
array([20, 29, 38, 47])
\end{lstlisting}
\end{frame}

\begin{frame}[fragile]
\frametitle{The {\tt NumPy} module}

\begin{lstlisting}[name=ex2]
>>> b**2
array([0, 1, 4, 9])
>>> 10*sin(a)
array([ 9.12945251, -9.88031624,  7.4511316 , -2.62374854])
>>> a<35
array([True, True, False, False], dtype=bool)
\end{lstlisting}

\pause
\begin{itemize}
\item Elementwise product vs matrix product:
\end{itemize}


\begin{lstlisting}[name=ex2]
>>> A = array( [[1,1],
...             [0,1]] )
>>> B = array( [[2,0],
...             [3,4]] )
>>> A*B                         # elementwise product
array([[2, 0],
       [0, 4]])
>>> dot(A,B)                    # matrix product
array([[5, 4],
       [3, 4]])
\end{lstlisting}
\end{frame}



\begin{frame}[fragile]
\frametitle{The {\tt NumPy} module}

\begin{itemize}
\item Array indexing and slicing:
\end{itemize}

\begin{lstlisting}[name=ex2]
>>> a
array([[ 0,  1,  2,  3,  4],
       [ 5,  6,  7,  8,  9],
       [10, 11, 12, 13, 14]])
>>> a[0]
array([0, 1, 2, 3, 4])
>>> a[1]
array([5, 6, 7, 8, 9])
>>> a[0:2]
array([[0, 1, 2, 3, 4],
       [5, 6, 7, 8, 9]])
>>>
>>> a[0, 0]
0
>>> a[1, 2]
7
>>> a[0:2, 0:2]
array([[0, 1],
       [5, 6]])
>>> 
>>> a[:,:]
array([[ 0,  1,  2,  3,  4],
       [ 5,  6,  7,  8,  9],
       [10, 11, 12, 13, 14]])
\end{lstlisting}
\end{frame}



\begin{frame}[fragile]
\frametitle{The {\tt NumPy} module}

\begin{lstlisting}[name=ex2]
>>> a[:, 0]
array([ 0,  5, 10])
>>> a[:, 0:1]
array([[ 0],
       [ 5],
       [10]])
>>> 
\end{lstlisting}

\pause
\begin{itemize}
\item Iterating over an array:
\end{itemize}

\begin{lstlisting}[name=ex2]
>>> for row in a:
...   print row
... 
[0 1 2 3 4]
[5 6 7 8 9]
[10 11 12 13 14]
>>> 
>>> for index in xrange(a.shape[1]):
...   print a[:, index]
... 
[ 0  5 10]
[ 1  6 11]
[ 2  7 12]
[ 3  8 13]
[ 4  9 14]
>>> 
>>> for elt in a.flat:
...   print elt,
... 
0 1 2 3 4 5 6 7 8 9 10 11 12 13 14
\end{lstlisting}
\end{frame}


\begin{frame}[fragile]
\frametitle{The {\tt NumPy} module}

\begin{itemize}
\item Array stacking:
\end{itemize}

\begin{lstlisting}[name=ex2]
>>> a = floor(10*random.random((2,2)))
>>> a
array([[ 1.,  1.],
       [ 5.,  8.]])
>>> b = floor(10*random.random((2,2)))
>>> b
array([[ 3.,  3.],
       [ 6.,  0.]])
>>> vstack((a,b))
array([[ 1.,  1.],
       [ 5.,  8.],
       [ 3.,  3.],
       [ 6.,  0.]])
>>> hstack((a,b))
array([[ 1.,  1.,  3.,  3.],
       [ 5.,  8.,  6.,  0.]])
\end{lstlisting}
\end{frame}


\begin{frame}[fragile]
\frametitle{The {\tt NumPy} module}

\begin{itemize}
\item Shallow copying:
\end{itemize}

\begin{lstlisting}[name=ex2]
>>> c = a.view()
>>> c == a
array([[ True,  True,  True,  True,  True],
       [ True,  True,  True,  True,  True],
       [ True,  True,  True,  True,  True]], dtype=bool)
>>> c is a
False
>>> a[0,0] = 1000
>>> a
array([[1000,    1,    2,    3,    4],
       [   5,    6,    7,    8,    9],
       [  10,   11,   12,   13,   14]])
>>> c
array([[1000,    1,    2,    3,    4],
       [   5,    6,    7,    8,    9],
       [  10,   11,   12,   13,   14]])
>>> 
\end{lstlisting}
\end{frame}






\begin{frame}[fragile]
\frametitle{The {\tt NumPy} module}
\scriptsize
\begin{itemize}
\item The default copy is the shallow copy:
\end{itemize}

\begin{lstlisting}[name=ex2]
>>> a
array([[1000,    1,    2,    3,    4],
       [   5,    6,    7,    8,    9],
       [  10,   11,   12,   13,   14]])
>>> b = a
>>> a[0,0] = 0
>>> b
array([[ 0,  1,  2,  3,  4],
       [ 5,  6,  7,  8,  9],
       [10, 11, 12, 13, 14]])
\end{lstlisting}

\pause
\begin{itemize}
\item Deep copying:
\end{itemize}

\begin{lstlisting}[name=ex2]
>>> a
array([[ 0,  1,  2,  3,  4],
       [ 5,  6,  7,  8,  9],
       [10, 11, 12, 13, 14]])
>>> b = a.copy()
>>> b[0,0] = 1000
>>> a
array([[ 0,  1,  2,  3,  4],
       [ 5,  6,  7,  8,  9],
       [10, 11, 12, 13, 14]])
>>> b
array([[1000,    1,    2,    3,    4],
       [   5,    6,    7,    8,    9],
       [  10,   11,   12,   13,   14]])
\end{lstlisting}
\end{frame}


\begin{frame}[fragile]
\frametitle{The {\tt NumPy} module}
\begin{itemize}
\item Logical arrays:
\end{itemize}

\begin{lstlisting}[name=ex2]
>>> a = arange(12).reshape(3,4)
>>> b = a > 4
>>> b              # b is a boolean with the same shape as a
array([[False, False, False, False],
       [False, True, True, True],
       [True, True, True, True]], dtype=bool)
>>> a[b]                                       # 1d array with the selected elements
array([ 5,  6,  7,  8,  9, 10, 11])
>>>
>>> a[b] = 0                                   # All elements of 'a' higher than 4 become 0
>>> a
array([[0, 1, 2, 3],
       [4, 0, 0, 0],
       [0, 0, 0, 0]])
\end{lstlisting}
\end{frame}




\begin{frame}[fragile]
\frametitle{The {\tt NumPy} module}

\begin{itemize}
\item Simple linear algebra:
\end{itemize}

\begin{lstlisting}[name=ex2]
>>> from numpy import *
>>> from numpy.linalg import *

>>> a = array([[1.0, 2.0], [3.0, 4.0]])
>>> print a
[[ 1.  2.]
 [ 3.  4.]]

>>> a.transpose()
array([[ 1.,  3.],
       [ 2.,  4.]])

>>> inv(a)
array([[-2. ,  1. ],
       [ 1.5, -0.5]])

>>> u = eye(2) # unit 2x2 matrix; "eye" represents "I"
>>> u
array([[ 1.,  0.],
       [ 0.,  1.]])
>>> j = array([[0.0, -1.0], [1.0, 0.0]])

>>> dot (j, j) # matrix product
array([[-1.,  0.],
       [ 0., -1.]])
\end{lstlisting}
\end{frame}

\begin{frame}[fragile]
\frametitle{The {\tt NumPy} module}

\begin{lstlisting}[name=ex2]
>>> trace(u)  # trace
2.0

>>> y = array([[5.], [7.]])
>>> solve(a, y)
array([[-3.],
       [ 4.]])

>>> eig(j)
(array([ 0.+1.j,  0.-1.j]),
array([[ 0.70710678+0.j,  0.70710678+0.j],
       [ 0.00000000-0.70710678j,  0.00000000+0.70710678j]]))
Parameters:
    square matrix

Returns
    The eigenvalues, each repeated according to its multiplicity.

    The normalized (unit "length") eigenvectors, such that the
    column ``v[:,i]`` is the eigenvector corresponding to the
    eigenvalue ``w[i]`` .
\end{lstlisting}
\end{frame}




\begin{frame}[fragile]
\frametitle{The {\tt NumPy} module}

\begin{itemize}
\item Matrices:
\end{itemize}

\begin{lstlisting}[name=ex2]
>>> A = matrix('1.0 2.0; 3.0 4.0')
>>> A
[[ 1.  2.]
 [ 3.  4.]]
>>> type(A)  # file where class is defined
<class 'numpy.matrixlib.defmatrix.matrix'>

>>> A.T  # transpose
[[ 1.  3.]
 [ 2.  4.]]

>>> X = matrix('5.0 7.0')
>>> Y = X.T
>>> Y
[[5.]
 [7.]]

>>> print A*Y  # matrix multiplication
[[19.]
 [43.]]

>>> print A.I  # inverse
[[-2.   1. ]
 [ 1.5 -0.5]]

>>> solve(A, Y)  # solving linear equation
matrix([[-3.],
        [ 4.]])
\end{lstlisting}
\end{frame}


\begin{frame}[fragile]
\frametitle{The {\tt NumPy} module}

\begin{itemize}
\item Caution: indexing and slicing are different between matrices and arrays.
\end{itemize}

\begin{lstlisting}[name=ex2]
>>> A = arange(12).reshape(3,4)
>>> M =  mat(A.copy())
>>> 
>>> print A[:,1]
[1 5 9]
>>> print M[:,1]
[[1]
 [5]
 [9]]
>>>
\end{lstlisting}
\end{frame}



\section{Other useful modules}
\begin{frame}
\frametitle{Other useful modules}
\scriptsize
\begin{itemize}
\item {\tt SciPy}: a module for scientific computing

\begin{center}
\begin{tabular}{l|l}
submodule & contents \\ \hline
{\tt cluster} & clustering algorithms \\
{\tt constants} & physical and mathematical constants \\
{\tt fftpack} & fast Fourier transform \\
{\tt integrate} & integration and ordinary differential equation solvers \\
{\tt interpolate} & interpolation and smoothing splines \\
{\tt io} & input and output \\
{\tt linalg} & linear algebra \\
{\tt ndimage} & N-dimensional image processing \\
{\tt odr} & orthogonal distance regression \\
{\tt optimize} & optimization and root-finding routines \\
{\tt signal} & signal processing \\
{\tt sparse} & sparse matrix routines \\
{\tt spatial}	 & spatial data structures and algorithms \\
{\tt special} & ``special" functions \\
{\tt stats} & statistical distributions and functions \\
{\tt weave} & integration with C/C++
\end{tabular}
\end{center}

\pause \item {\tt matplotlib}: graphics and plotting
\end{itemize}
\end{frame}

\begin{frame}
\frametitle{Other useful modules}
\begin{itemize}
\item {\tt PyCUDA}: writing and executing GPU kernels from within Python.
\begin{itemize}
\pause \item Stay tuned $\ldots$
\end{itemize}
\end{itemize}
\end{frame}


\begin{frame}
\frametitle{Outline}
\tableofcontents
\end{frame}

\begin{frame}
\frametitle{Resources} \small

\begin{itemize}
\item Guides:
\begin{enumerate}
\item David M. Beazley. \emph{Python Essential Reference: Fourth Edition}. Addison-Wesley, 2009.
\pause \item Tentative NumPy Tutorial. \url{http://www.scipy.org/Tentative\_NumPy\_Tutorial}.
\pause \item SciPy Tutorial. \url{http://docs.scipy.org/doc/scipy/reference/tutorial/general.html}
\item Matplotlib. \url{http://matplotlib.org/}.
\end{enumerate}
\pause \item Code from today:
\begin{itemize}
\item \href{http://will-landau.com/gpu/Code/Python/Basic_Python/IntroPython.py}{IntroPython.py}
\end{itemize}
\end{itemize}
\end{frame}


\begin{frame}
\frametitle{That's all for today.}
\begin{itemize}
\item Series materials are available at \url{http://will-landau.com/gpu}.
\end{itemize}
\end{frame}



\end{document}