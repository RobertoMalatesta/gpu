\documentclass[handout]{beamer}
\usetheme{Marburg}
\useoutertheme{infolines}
\newcommand{\answers}{1}

\usepackage{amsmath}
\usepackage{caption}
\usepackage{color}
\usepackage{enumerate}
\usepackage{listings}
\usepackage{hyperref}
\usepackage{mathrsfs}
\usepackage{natbib}
\usepackage{pgffor}
\usepackage{url}

\providecommand{\all}{\ \forall \ }
\providecommand{\bs}{\backslash}
\providecommand{\e}{\varepsilon}
\providecommand{\E}{\ \exists \ }
\providecommand{\lm}[2]{\lim_{#1 \rightarrow #2}}
\providecommand{\m}[1]{\mathbb{#1}}
\providecommand{\nv}{{}^{-1}}
\providecommand{\ov}[1]{\overline{#1}}
\providecommand{\p}{\newpage}
\providecommand{\q}{$\quad$ \newline}
\providecommand{\rt}{\rightarrow}
\providecommand{\Rt}{\Rightarrow}
\providecommand{\vc}[1]{\boldsymbol{#1}}
\providecommand{\wh}[1]{\widehat{#1}}

\hypersetup{colorlinks,linkcolor=,urlcolor=blue}
\numberwithin{equation}{section}

\definecolor{dkgreen}{rgb}{0,0.6,0}
\definecolor{gray}{rgb}{0.5,0.5,0.5}
\definecolor{mauve}{rgb}{0.58,0,0.82}

\lstset{ 
  language=C,                % the language of the code
  basicstyle= \footnotesize,           % the size of the fonts that are used for the code
  numbers=left,
  numberfirstline=true,
  numbersep=5pt,                  % how far the line-numbers are from the code
  backgroundcolor=\color{white},      % choose the background color. You must add \usepackage{color}
  showspaces=false,               % show spaces adding particular underscores
  showstringspaces=false,         % underline spaces within strings
  showtabs=false,                 % show tabs within strings adding particular underscores
  frame=lrb,                   % adds a frame around the code
  rulecolor=\color{black},        % if not set, the frame-color may be changed on line-breaks within not-black text 
  tabsize=2,                      % sets default tabsize to 2 spaces
  captionpos=t,                   % sets the caption-position 
  breaklines=true,                % sets automatic line breaking
  breakatwhitespace=false,        % sets if automatic breaks should only happen at whitespace
  %title=\lstname,                   % show the filename of files included with \lstinputlisting;
  keywordstyle=\color{blue},          % keyword style
  commentstyle=\color{gray},       % comment style
  stringstyle=\color{dkgreen},         % string literal style
  escapeinside={\%*}{*)},            % if you want to add LaTeX within your code
  morekeywords={*, ...},               % if you want to add more keywords to the set
  xleftmargin=0.053in, % left horizontal offset of caption box
  xrightmargin=-.03in % right horizontal offset of caption box
}

\DeclareCaptionFont{white}{\color{white}}
\DeclareCaptionFormat{listing}{\parbox{\textwidth}{\colorbox{gray}{\parbox{\textwidth}{#1#2#3}}\vskip-0.05in}}
\captionsetup[lstlisting]{format = listing, labelfont = white, textfont = white}
%For caption-free listings, comment out the 3 lines above and uncomment the 2 lines below.
% \captionsetup{labelformat = empty, labelsep = none}
 %\lstset{frame = single}

\title{Pointers and dynamic allocation in C}
\author{Will Landau}
\date{\today}
\institute{Iowa State University}

\begin{document}

\begin{frame}
\titlepage
 \end{frame}
 
\begin{frame}
\frametitle{Outline}
\tableofcontents
\end{frame} 
 
 \AtBeginSection[]
{
   \begin{frame}
       \frametitle{Outline}
       \tableofcontents[currentsection]
   \end{frame}
}

\section{Computer memory}

\begin{frame}
\frametitle{Computer memory}
\begin{itemize}
\item Fundamentally, all data is encoded in \emph{byte code}, strings of ones and zeros.

\begin{align*}
0100101100101100101\cdots
\end{align*}

\pause \item {\bf Bit}: a 1 or 0 in byte code.
\pause \item {\bf Byte}: a string of 8 bits. For example, 00110100.
\pause \item {\bf Word}: 
\begin{itemize}
\item a natural unit of data, the length of which depends on the processor.
\pause \item On ``32-bit architectures", a word is a string of 32 bits (4 bytes).
\end{itemize}
\end{itemize}
\end{frame}


\begin{frame}
\frametitle{Compute memory}

\begin{itemize}
\item Computer memory is a linear array of bytes. Each byte has a word-sized index called an \emph{address}, or \emph{pointer}.
\end{itemize}

\pause \begin{center}
\begin{tabular}{c|c|c}
Address & Stored Value & Variable Name  \\ \hline 
24399440  & 3  & a \\ 
24399441  & & \\ 
24399442  &  &\\
24399443  &  & \\ \hline
24399444  &  6.43451 & b\\ 
24399445  & & \\ 
24399446  &  &\\ 
24399447  & & \\ \hline
$\vdots$ & $\vdots$
\end{tabular}
\end{center}

\begin{itemize}
\pause \item Note: we use the address, 24399440 (not 24399441 or 24399442) to refer to the storage space of variable {\tt a}.
\end{itemize}
\end{frame}

\begin{frame}
\frametitle{Computer memory}
\begin{itemize}
\item I condense the previous table and write:
\end{itemize}

\begin{center}
\begin{tabular}{c|c|c}
Address & Stored Value & Variable Name  \\ \hline 
24399440  & 3  & a \\ \hline
24399444  &  6.43451 & b \\ \hline
$\vdots$ & $\vdots$
\end{tabular}
\end{center}


\begin{itemize}
\pause \item We say that:
\begin{itemize}
\pause \item 24399440 is the address of variable {\tt a}.
\pause \item 3 is the stored value at the address, 24399440.
\pause \item {\tt a} is the variable pointed to by 24399440.
\pause \item 3 is the value pointed to by 24399440. 
\end{itemize}
\end{itemize}
\end{frame}



\section{Pointers}

\begin{frame}
\frametitle{Declaring pointer variables}

\begin{itemize}
\item Examples:
\begin{itemize}
\pause \item Write {\tt int *pa;} to declare an int pointer variable: a variable whose value is the address of an integer.
\pause \item Write {\tt float *pa;} to declare a float pointer variable: a variable whose value is the address of a float.
\pause \item Write {\tt double *pa;} to declare a double pointer variable: a variable whose value is the address of a double.
\end{itemize}
\pause \item The type of a pointer variable depends on the data type pointed to because: 
\begin{itemize}
\pause \item Different data types take up different amounts of memory.
\pause \item The computer needs to know how to interpret the bytes of memory stored. Ints and floats, for example, are encoded differently.
\end{itemize}
\end{itemize}

\end{frame}


\begin{frame}[fragile]
\frametitle{Example}

\lstinputlisting[title=http://will-landau.com/gpu/Code/C/pointers/ex0.c]{../../Code/C/pointers/ex0.c}

\lstset{ title=output}
\pause \begin{lstlisting}
> gcc ex0.c -o ex0
> ./ex0
a = 17
a = 0.000000
\end{lstlisting}
\end{frame}


\begin{frame}[fragile] \small

\lstset{ title=output, basicstyle=\footnotesize}

\lstinputlisting[title=http://will-landau.com/gpu/Code/C/pointers/ex1.c]{../../Code/C/pointers/ex1.c}

\pause \begin{lstlisting}
> gcc ex1.c -o ex1
> ./ex1
a = 0
&a = 1355533180
\end{lstlisting}


\pause \begin{center}
\begin{tabular}{c|c|c} \scriptsize
Variable & Address & Stored value \\ \hline
a &1355533180 & 3
\end{tabular}
\end{center}
\end{frame}

\lstset{ title=output}

\begin{frame}
\begin{itemize}
\item Let {\tt a} be an int and {\tt pa} be a pointer to an int. Then:
\begin{itemize} \small
\pause \item {\tt \&a} returns the address of {\tt a} (referencing). 
\pause \item {\tt *pa} returns the value pointed to by {\tt a} (dereferencing).
\end{itemize}
\end{itemize}

\pause \lstinputlisting[title=http://will-landau.com/gpu/Code/C/pointers/ex2.c]{../../Code/C/pointers/ex2.c}
\end{frame}

\begin{frame}[fragile]
\begin{lstlisting}
> gcc ex2.c -o ex2
> ./ex2
a=1
&a = 1420507900 
?pa = 1
pa = 1420507900
&pa = 1420507888
\end{lstlisting}

\begin{center}
\pause \begin{tabular}{c|c|c}
Variable & Address & Stored value \\ \hline
a & 1420507900 & 1 \\ \hline
pa & 1420507888 & 1420507900
\end{tabular}
\end{center}
\end{frame}


\begin{frame}
\pause \lstinputlisting[title=http://will-landau.com/gpu/Code/C/pointers/ex3.c]{../../Code/C/pointers/ex3.c}
\end{frame}

\begin{frame}[fragile]
\begin{lstlisting}
> gcc ex3.c ?o ex3 
> ./ex3
a=0
&a = 1537735420 b=1
&b = 1537735416 ?pa = 1
pa = 1537735416
&pa = 1537735408
\end{lstlisting}

\begin{center}
\pause \begin{tabular}{c|c|c}
Variable & Address & Stored value \\ \hline
a & 1537735420 & 0 \\ \hline
b & 1537735416 & 1 \\ \hline
pa & 1537735408 & 1537735416
\end{tabular}
\end{center}
\end{frame}

\section{Passing arguments by value and by reference}


\begin{frame}
\frametitle{Passing by value}
\lstinputlisting[title=http://will-landau.com/gpu/Code/C/pointers/ex4.c]{../../Code/C/pointers/ex4.c}
\end{frame}

\begin{frame}[fragile]
\frametitle{Passing by value}

\begin{lstlisting}
> gcc ex4.c -o ex4
> ./ex4
a = 0
\end{lstlisting}

\begin{itemize}
\pause \item {\tt a } was passed to {\tt fun()} by \emph{value}
\pause \item {\tt fun()} received a local copy of {\tt a} and then lost it when the function call terminated.
\pause \item The copy of {\tt a} in {\tt int main()} remained unchanged.
\end{itemize}
\end{frame}




\begin{frame}
\frametitle{Passing by reference}
\lstinputlisting[title=http://will-landau.com/gpu/Code/C/pointers/ex5.c]{../../Code/C/pointers/ex5.c}
\end{frame}

\begin{frame}[fragile]
\frametitle{Passing by reference}

\begin{lstlisting}
> gcc ex5.c -o ex5
> ./ex5
a = 1
\end{lstlisting}

\begin{itemize}
\pause \item {\tt a } was passed to {\tt fun()} by \emph{reference}
\pause \item {\tt fun()} received a local copy of a \emph{pointer} to {\tt a} in {\tt int main()}.
\pause \item When {\tt fun()} terminated, it lost its copy of the address of {\tt a}, but it did not have an actual copy of {\tt a} to lose.
\end{itemize}
\end{frame}


\begin{frame}
\lstinputlisting[title=http://will-landau.com/gpu/Code/C/pointers/ex6.c]{../../Code/C/pointers/ex6.c}
\end{frame}

\begin{frame}[fragile]

\begin{lstlisting}
> gcc ex6.c -o ex6
> ./ex6
a = 0
*pa = 1
\end{lstlisting}

\begin{itemize}
\pause \item {\tt pa} id not contain the address of {\tt a}, so {\tt a} was not passed at all.
\end{itemize}
\end{frame}


\begin{frame}
\lstinputlisting[title=http://will-landau.com/gpu/Code/C/pointers/ex7.c]{../../Code/C/pointers/ex7.c}
\end{frame}

\begin{frame}[fragile]

\begin{lstlisting}
> gcc ex7.c -o ex7
> ./ex7
a = 1
*pa = 1
\end{lstlisting}

\begin{itemize}
\pause \item Since {\tt pa} points to {\tt a} and {\tt pa} was passed by value, {\tt a} was passed by reference.
\end{itemize}
\end{frame}


\begin{frame}[fragile]
\frametitle{Caution}
\begin{itemize}
\item Assign values to pointers before dereferencing them.
\end{itemize}

\pause \lstinputlisting[title=http://will-landau.com/gpu/Code/C/pointers/ex7.c]{../../Code/C/pointers/caution1.c}
\pause 
\begin{lstlisting}
> gcc caution1 . c ?o caution1
> ./caution1
Bus error: 10
\end{lstlisting}

\begin{itemize}
\pause \item The value of {\tt a} is some garbage number that isn't a real address! It points to nowhere!
\end{itemize}
\end{frame}


\section{Arrays}


\begin{frame}[fragile]
\frametitle{Arrays}
\lstinputlisting[title=http://will-landau.com/gpu/Code/C/pointers/ar1.c]{../../Code/C/pointers/ar1.c}

\pause \begin{lstlisting}
> gcc ar1.c -o ar1 
> ./ar1
1
23
17
\end{lstlisting}
\end{frame}



\begin{frame}[fragile]
\frametitle{Arrays}
\lstinputlisting[title=http://will-landau.com/gpu/Code/C/pointers/ar2.c]{../../Code/C/pointers/ar2.c}
\end{frame}

\begin{frame}[fragile]
\frametitle{Arrays}
\begin{lstlisting}
> gcc ar2.c -o ar2
> ./ar2
9
17
25
7
\end{lstlisting}
\end{frame}



\begin{frame}[fragile]
\frametitle{Arrays}
\lstinputlisting[title=http://will-landau.com/gpu/Code/C/pointers/ar3.c, basicstyle=\tiny]{../../Code/C/pointers/ar3.c}
\end{frame}

\begin{frame}[fragile]
\frametitle{Arrays}
\begin{lstlisting}
> gcc ar3.c -o ar3
> ./ar3
9
17
25
7
1518070576
\end{lstlisting}

\small
\begin{itemize}
\pause \item {\tt pa} is actually pointer to the first element of the array.
\end{itemize}

\begin{center}
\pause \begin{tabular}{l|c|c}
Variable name(s) & Address & Stored value \\ \hline
pa & & \\ \hline
pa[0], *pa & 1518070576 & 9 \\ \hline
pa[1], *(pa + 1) & 1518070580 & 17 \\ \hline
pa[2], *(pa + 2) &1518070584 & 25 \\ \hline
pa[3], *(pa + 3) & 1518070588 & 7
\end{tabular}
\end{center}
\end{frame}


\begin{frame}[fragile]
\frametitle{Caution} \small

\begin{itemize}
\item Every (statically allocated) array has a set length. Do not dereference beyond this length. 
\pause \item C lets you, but you risk a:
\pause \item bus error: dereferencing an address that points to nothing.
\pause \item segmentation fault: dereferencing an address that exists but that the program does not have permission to dereference (out of bounds).
\end{itemize}

\pause \lstinputlisting[title=http://will-landau.com/gpu/Code/C/pointers/caution2.c, basicstyle=\tiny]{../../Code/C/pointers/caution2.c}
\end{frame}

\begin{frame}[fragile]
\frametitle{Caution} 
\begin{lstlisting}
> gcc caution2.c -o caution2
> ./caution2
Segmentation fault: 11
\end{lstlisting}
\end{frame}

\section{Dynamic memory allocation}

\begin{frame}
\frametitle{Dynamic memory allocation}

\begin{itemize}
\item {\bf Static memory allocation}: acquiring a fixed-sized piece of memory for a variable at compile time.
\pause \item {\bf Dynamic memory allocation}: acquiring a variable-length piece of memory at runtime. 
\pause \item To use dynamic memory,
\begin{enumerate}
\pause \item use {\tt malloc()}, defined in {\tt stdlib.h}, to allocate memory.
\pause \item use the variable like an ordinary array.
\pause \item use {\tt free()} to release the memory. 
\end{enumerate}
\end{itemize}
\end{frame}


\begin{frame}
\frametitle{Dynamic memory allocation}
\lstinputlisting[title=http://will-landau.com/gpu/Code/C/pointers/dy1.c, basicstyle=\tiny]{../../Code/C/pointers/dy1.c}
\end{frame}

\begin{frame}[fragile]
\frametitle{Dynamic memory allocation}
\begin{lstlisting}
> gcc dy1.c -o dy
> ./dy
a[0] = 10
a[1] = 11
a[2] = 14
a[3] = 19
a[4] = 26
a[5] = 35
a[6] = 46
a[7] = 59
a[8] = 74
a[9] = 91
\end{lstlisting}
\end{frame}



\begin{frame}[fragile]
\frametitle{Dynamic memory allocation}

\lstset{title=http://will-landau.com/gpu/Code/C/pointers/dy2.c, basicstyle=\tiny}
\begin{lstlisting}
#include <stdio.h>
#include <stdlib.h>

#define M 10
#define N 15

void fill(float *x, int size){
  int i;
  for(i = 0; i < size; ++i){
    x[i] = 10.25 + i*i;
  }
}
\end{lstlisting}
\end{frame}

\begin{frame}[fragile]
\frametitle{Dynamic memory allocation}

\lstset{title=http://will-landau.com/gpu/Code/C/pointers/dy2.c, basicstyle=\tiny}
\begin{lstlisting}
int main(){
  int i;
  float *a, *b;
  
  a = (float *) malloc(M * sizeof(float));
  b = (float *) malloc(N * sizeof(float));
  
  fill(a, M);
  fill(b, N);
  
 for(i = 0; i < M; ++i){
    printf("a[%d] = %f\n", i, a[i]);
  }
  printf("\n");
  
  for(i = 0; i < N; ++i){
    printf("b[%d] = %f\n", i, b[i]);
  }
  
  free(a);
  free(b);
}
\end{lstlisting}
\end{frame}


\begin{frame}[fragile]
\frametitle{Dynamic memory allocation}
\begin{lstlisting}
> gcc dy2.c -o dy2
> ./dy2
a[0] = 10.250000
a[1] = 11.250000
a[2] = 14.250000
a[3] = 19.250000
a[4] = 26.250000
a[5] = 35.250000
a[6] = 46.250000
a[7] = 59.250000
a[8] = 74.250000
a[9] = 91.250000
\end{lstlisting}
\end{frame}


\begin{frame}[fragile]
\frametitle{Dynamic memory allocation}
\begin{lstlisting}

b[0] = 10.250000
b[1] = 11.250000
b[2] = 14.250000
b[3] = 19.250000
b[4] = 26.250000
b[5] = 35.250000
b[6] = 46.250000
b[7] = 59.250000
b[8] = 74.250000
b[9] = 91.250000
b[10] = 110.250000
b[11] = 131.250000
b[12] = 154.250000
b[13] = 179.250000
b[14] = 206.250000
\end{lstlisting}
\end{frame}

\begin{frame}
\frametitle{Outline}
\tableofcontents
\end{frame}

\begin{frame}
\frametitle{Resources}
\begin{enumerate}
\item Kernighan, B. W., and Ritchie, D. M. �The ANSI C Programming Language�. 2nd Ed.
\pause \item Savitch, W. �Absolute C++�. 3rd Ed.
\pause \item Jensen, T. �A Tutorial on Pointers and Arrays in C�. http://pw1.netcom.com/~tjensen/ptr/pointers.htm
\pause \item GPU series materials: \url{http://will-landau.com/gpu}.
\end{enumerate}
\end{frame}

\end{document}